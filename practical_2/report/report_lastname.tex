\documentclass{article}

% if you need to pass options to natbib, use, e.g.:
% \PassOptionsToPackage{numbers, compress}{natbib}
% before loading nips_2016
%
% to avoid loading the natbib package, add option nonatbib:
% \usepackage[nonatbib]{nips_2016}

\usepackage[final]{nips_2016}

% to compile a camera-ready version, add the [final] option, e.g.:
% \usepackage[final]{nips_2016}

\usepackage[utf8]{inputenc} % allow utf-8 input
\usepackage[T1]{fontenc}    % use 8-bit T1 fonts
\usepackage{hyperref}       % hyperlinks
\usepackage{url}            % simple URL typesetting
\usepackage{booktabs}       % professional-quality tables
\usepackage{amsfonts}       % blackboard math symbols
\usepackage{nicefrac}       % compact symbols for 1/2, etc.
\usepackage{microtype}      % microtypography

\title{Report for the Deep Learning Course Assignment 2 }

% The \author macro works with any number of authors. There are two
% commands used to separate the names and addresses of multiple
% authors: \And and \AND.
%
% Using \And between authors leaves it to LaTeX to determine where to
% break the lines. Using \AND forces a line break at that point. So,
% if LaTeX puts 3 of 4 authors names on the first line, and the last
% on the second line, try using \AND instead of \And before the third
% author name.

\author{
  David S.~Hippocampus \\
  \texttt{hippo@cs.cranberry-lemon.edu}
}

\begin{document}
% \nipsfinalcopy is no longer used

\maketitle

\begin{abstract}
  Should contain information about the current task and the summary of the study
  of the MLP model on CIFAR10 dataset.
\end{abstract}

\section{Task 1}
Should contain information about the current task and some description of the MLP model and TensorFlow framework.
Put the answers for the questions from the Task 1 into this section.

\section{Task 2}
Should contain your study of the default model from the Task 2.

\section{Task 3}
Should contain results of your experiments.
Please, describe all experiments settings to make the report self-contained without this file.
Put each experiment in separate subsection.

\subsection{Experiment-1: Weight initialization}
\subsection{Experiment-2: Interaction between initialization and activation}
\subsection{Experiment-3: Architecture}
\subsection{Experiment-4: Optimizers}
\subsection{Experiment-5: Be creative}

\section{Conclusion}
Should contain conclusion of this study.
For example, you can try to answer the following questions.
What was done during this assignment? What features of TensorBoard were positive and what were negative for implementing MLP model and performing the experiments?
What are the main insights you got from the study of the MLP model on CIFAR10 dataset?

\section*{References}

\small

[1] Alexander, J.A.\ \& Mozer, M.C.\ (1995) Template-based algorithms
for connectionist rule extraction. In G.\ Tesauro, D.S.\ Touretzky and
T.K.\ Leen (eds.), {\it Advances in Neural Information Processing
  Systems 7}, pp.\ 609--616. Cambridge, MA: MIT Press.

[2] Bower, J.M.\ \& Beeman, D.\ (1995) {\it The Book of GENESIS:
  Exploring Realistic Neural Models with the GEneral NEural SImulation
  System.}  New York: TELOS/Springer--Verlag.

[3] Hasselmo, M.E., Schnell, E.\ \& Barkai, E.\ (1995) Dynamics of
learning and recall at excitatory recurrent synapses and cholinergic
modulation in rat hippocampal region CA3. {\it Journal of
  Neuroscience} {\bf 15}(7):5249-5262.

\end{document}
